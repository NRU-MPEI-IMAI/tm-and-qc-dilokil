\documentclass{article}
\usepackage[utf8]{inputenc}
\usepackage[T2A]{fontenc}
\usepackage[utf8]{inputenc}
\usepackage[russian]{babel}
\usepackage[margin=3cm]{geometry}
\usepackage{paralist}
\usepackage{amsthm, amsmath, amsfonts, amssymb}
\usepackage{mathtools} % \mathclap
\usepackage{bm}
\usepackage{dsfont}
\usepackage{hyperref}
\usepackage{graphicx}
\usepackage{multirow}
\usepackage{comment}
\usepackage{xcolor, colortbl}
\usepackage{xifthen, xspace}
\usepackage{caption, subcaption}
\usepackage{lscape}
\usepackage{braket}
\usepackage{epigraph}
\usepackage{sectsty}
\usepackage{listings}

\hypersetup{
    colorlinks=true,
    linkcolor=blue,
    filecolor=magenta,      
    urlcolor=blue,
    pdftitle={Overleaf Example},
    pdfpagemode=FullScreen,
    }

\title{Теоретические модели вычислений \\
ДЗ №3: Машины Тьюринга и квантовые вычисления}
\author{Дмитрий Пугачёв А-05-19}
\date{May 2022}

\begin{document}

\maketitle


\section{Машины Тьюринга}

\subsection{Операции с числами}

Реализуйте машины Тьюринга, которые позволяют выполнять следующие операции:
\begin{enumerate}
    \item Сложение двух унарных чисел\\
    Алгоритм для решения этой задачи предельно прост:\\
    Для ее решения понадобилось 3 ключевых состояния, первое для поиска знака "+"\; и замены его на 1 (элемент унарной системы счисления), затем состояние, сдвигающее головку до конца слова и завершающее состояние, которое удаляет последний элемент, так как он "переносится" на место знака сложения.\\
    Код находится по адресу "Task1/1\_1.yml".\\
    \item Умножение унарных чисел\\
    Алгоритм для этой задачи работает следующим образом: \\
    Сперва мы совершаем проход по первому числу, пока не встретим 1 (как и в прошлой задаче это элемент унарной системы). Затем для каждой 1 во втором числе мы делаем ее копию после исходного выражения. Когда второе число закончится (встретим пробельный символ) - возвращаемся в начало и проделываем то же самое для оставшихся цифр в числе. В конце всего алгоритма исходное выражение удаляется.\\
    Код находится по адресу "Task1/1\_2.yml".\\
\end{enumerate}


\subsection{Операции с языками и символами}

Реализуйте машины Тьюринга, которые позволяют выполнять следующие операции:
\begin{enumerate}
    \item Принадлежность к языку $L = \{ 0^n1^n2^n \}, n \ge 0$ \\
    Ключевыми компонентами алгоритма для этой задачи являются 3 состояния, которые поочередно проверяют наличие одинакового количества 0, 1 и 2. Достигается это путем зацикленного прохода по всему слову, причем при каждом проходе первые символы алфавита заменяются на a, b, c соответственно. Цикл завершает свою работу, если во время выполнения какой-либо вершины встречаем неожиданный элемент.
    Результатом программы является напечатанная буква "F" или "T" означающая отсутвие принадлежности или ее присутствие.\\
    Код находится по адресу "Task2/2\_2.yml".\\
    \item Проверка соблюдения правильности скобок в строке (минимум 3 вида скобок) 
    Следующий алгоритм реализует похожую идею, но абсолютно в другом порядке:\\
    Происходит движение головки в сторону конца входного слова, при этом - если встречается какая либо закрывающая скобка, то алгоритм ставит на ее место символ, говорящий о том, что эта скобка уже рассматривалась, и затем головка меняет направление для поиска пары для нужной скобочки. Если все скобки обладают парами, то программа выведет "T", иначе "F".\\
    Код находится по адресу "Task2/2\_2.yml".\\
    \item Поиск минимального по длине слова в строке (слова состоят из символов 1 и 0 и разделены пробелом)\\
    Идея этого алгоритма основана на рассуждениях из прошлой задачи. Имеется набор слов разной длины, если поочередно у каждого слова заменять символ (0 или 1) на букву (a или b) начиная с первого и заканчивая номером последнего символа в минимальном слове, то мы получим предложение, где только одно слово полностью состоит из букв. Такой результат достигается постепенной заменой всех цифр на буквы и при обратном ходе головки, проверками последнего элемента в слове. Если последний элемент буква, то буквы заменяются обратно на цифры и головка устанавливается на начало этого слова. Особенностью этого алгоритма является установка головки на последнее минимальное слово в предложении.\\
    Код находится по адресу "Task2/2\_3.yml".\\
\end{enumerate}


\section{Квантовые вычисления}


\subsection{Генерация суперпозиций 1 (1 балл)}

Дано $N$ кубитов ($1 \le N \le 8$) в нулевом состоянии $\Ket{0\dots0}$. Также дана некоторая последовательность битов, которое задаёт ненулевое базисное состояние размера $N$. Задача получить суперпозицию нулевого состояния и заданного.

$$\Ket{S} = \frac{1}{\sqrt2}(\Ket{0\dots0} +\Ket{\psi})$$

То есть требуется реализовать операцию, которая принимает на вход:

\begin{enumerate}
    \item Массив кубитов $q_s$
    \item Массив битов $bits$ описывающих некоторое состояние $\Ket{\psi}$. Это массив имеет тот же самый размер, что и $qs$. Первый элемент этого массива равен $1$.
\end{enumerate}
Решение:
\begin{lstlisting}
namespace Solution {
    open Microsoft.Quantum.Primitive;
    open Microsoft.Quantum.Canon;
    operation Solve(qs: Qubit[], bits: Bool[]) : () {
        body { 
            H(qs[0]);
            for (i in 1..Length(qs) - 1) {
                if (bits[i]) {
                    CNOT(qs[0], qs[i]); 
                } 
            }                  
        }
    }
}
\end{lstlisting}


\subsection{Различение состояний 1 (1 балл)}

Дано $N$ кубитов ($1 \le N \le 8$), которые могут быть в одном из двух состояний:

$$\Ket{GHZ} = \frac{1}{\sqrt2}(\Ket{0\dots0} +\Ket{1\dots1})$$
$$\Ket{W} = \frac{1}{\sqrt N}(\Ket{10\dots00}+\Ket{01\dots00} + \dots +\Ket{00\dots01})$$

Требуется выполнить необходимые преобразования, чтобы точно различить эти два состояния. Возвращать $0$, если первое состояние и 1, если второе. 
\\\\
Решение:
\begin{lstlisting}
namespace Solution {
    open Microsoft.Quantum.Primitive;
    open Microsoft.Quantum.Canon;
    operation Solve(qs: Qubit[]) : Int {
        body {   
            mutable one = 0;
            for (q in qs) {
                if (M(q) == One) { 
                    set one = one + 1; 
                }
            }
            if (one == 1) {
                return 1;
            } else {
                return 0;
            }                
        }
    }
}
\end{lstlisting}

\end{document}